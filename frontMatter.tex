\RequirePackage{fix-cm}

\documentclass{svjour3}                      % onecolumn (standard format)
%\documentclass[smallcondensed]{svjour3}     % onecolumn (ditto)
%\documentclass[smallextended]{svjour3}      % onecolumn (second format)
%\documentclass[twocolumn]{svjour3}          % twocolumn


\smartqed  % flush right qed marks, e.g. at end of proof

\usepackage[hmargin=3.4cm,vmargin=3cm]{geometry} % This one is specific for springer journals. Define a smaller margin
\usepackage{amssymb}
\usepackage{amsmath}
\usepackage{graphicx}
\usepackage[retainorgcmds]{IEEEtrantools}
\usepackage[colorlinks]{hyperref}
\usepackage{siunitx}
\usepackage{pdflscape}
% disable labelindent
\let\labelindent\relax
\usepackage{enumitem}
\usepackage[normalem]{ulem}
\useunder{\uline}{\ul}{}
\usepackage{setspace} % \doublespacing
\usepackage[utf8]{inputenc}
\usepackage[linesnumbered, ruled, vlined]{algorithm2e}
\usepackage{rotating}
% \usepackage{subfigure}
\usepackage{nicefrac}
\usepackage{todonotes}
\usepackage{hyperref}
\usepackage{xspace}
\usepackage{float}
\usepackage{caption}
\usepackage{subcaption}
% \usepackage{floatrow}
\usepackage{rotating, lscape, longtable, tabu, booktabs, siunitx, multirow}
%\usepackage[capitalise, noabbrev]{cleveref}
\usepackage{tkz-graph}
\usepackage{verbatim}
\usepackage{mathtools}


\newcommand{\uset}[1]{\ifmmode\left\{\,#1\,\right\}\else\{\,#1\,\}\fi}
\newcommand{\ulst}[1]{\ifmmode\left[\,#1\,\right]\else[\,#1\,]\fi}
\newcommand{\upar}[1]{\ifmmode\left(\,#1\,\right)\else(\,#1\,)\fi}
\newcommand{\uioc}[1]{\ifmmode\left(\,#1\,\right]\else(\,#1\,]\fi}
\newcommand{\uico}[1]{\ifmmode\left[\,#1\,\right)\else[\,#1\,)\fi}

\newcommand{\ie}{i.\,e.,\xspace}
\newcommand{\eg}{e.\,g.,\xspace}
\newcommand{\false}{\texttt{false}}
\newcommand{\true}{\texttt{true}}

% tikz libraries
\usepackage{tkz-graph}
\usepackage{verbatim}
\usetikzlibrary{arrows,arrows.meta,shapes,decorations.pathmorphing, decorations.markings, positioning}
\let\svtikzpicture\tikzpicture
\def\tikzpicture{\noindent\svtikzpicture}


\begin{document}

\title{Article title \thanks{This study was financed in part by the \emph{Coordenação de Aperfeiçoamento de Pessoal de Nível Superior - Brasil} (CAPES) - Finance Code 001, the \emph{Conselho Nacional de Desenvolvimento Científico e Tecnológico - Brasil} (CNPq), and the \emph{Fundação de Amparo à Pesquisa de Minas Gerais - Brasil} (FAPEMIG)}}.
%\subtitle{Do you have a subtitle?\\ If so, write it here}

%\titlerunning{Short form of title}        % if too long for running head

\author{Iago A. Carvalho \and Homer J. Simson}

%\authorrunning{Short form of author list} % if too long for running head

\institute{Iago A. Carvalho \at
            Department of Computer Science, Universidade Federal de Minas Gerais, Av. Antônio Carlos 6627, Belo Horizonte, MG 31270-010, Brazil \\
            Tel.: +55-31-971216117\\
           \email{\href{mailto:iagoac@dcc.ufmg.br}{iagoac@dcc.ufmg.br}} \\            
		   \and           
            Homer J. Simpson \at 
            Fox
}

\date{Received: xxxx / Accepted: yyyy}

\maketitle

\begin{abstract}
There will be an abstract here.
\keywords{Keyword 1 \and Keyword 2 \and Keyword 3 \and Keyword 4 \and Keyword 5}
% \PACS{PACS code1 \and PACS code2 \and more}
% \subclass{MSC code1 \and MSC code2 \and more}
\end{abstract}

\section{Introduction} \label{sec:intro}

% Contributions of the paper
In this paper, we first...

% Organization of the paper
The remainder of this paper is organized as follows.
Section~\ref{sec:problem} formally defines our problem.
Section~\ref{sec:related} reviews the related works.
Section~\ref{sec:algorithms} presents the algorithms, which are evaluated on Section~\ref{sec:experiments}. Finally, Section~\ref{sec:conclusions} draws the concluding remarks of our paper.

\section{The problem} \label{sec:problem}


\section{Related works} \label{sec:related}


\section{Algorithms} \label{sec:algorithms}

\subsection{Algorithm 1}

\subsection{Algorithm 2}

\subsection{Algorithm n}


\section{Computational experiments} \label{sec:experiments}

% Computational envoirnment description
The computational experiments have been done on a single core of an Intel
with x.x Ghz clock and x Gb of RAM,
unning underthe  operating  system  Linux  Ubuntu.
We used the ILOG  CPLEX  solver  version  12.8  with default parameters setting. 
The algorithms were implemented in C{}\verb!++! along with the ILOG Concert Technology and compiled with the GNU g{}\verb!++! 8.2.0. 
The running time of all algorithms has been set to 7200 seconds.


\subsection{Instance's description} \label{subsec:instances}


\subsection{Expriment 1}

\subsection{Experiment 2}

\subsection{Experiment m}


\section{Conclusions} \label{sec:conclusions}


% BibTeX users please use one of
%\bibliographystyle{spbasic}      % basic style, author-year citations
\bibliographystyle{spmpsci}      % mathematics and physical sciences
%\bibliographystyle{spphys}       % APS-like style for physics
\bibliography{bibsample}   % name your BibTeX data base

\end{document}
% end of file template.tex
